% PRÓLOGO

\textbf{1935, Detroit. Walter Pitts corre pela rua, perseguido por valentões.}

Ele se esconde na biblioteca pública para se proteger, e ele se esconde tão bem que a equipe da biblioteca nem percebe que ele está lá, e eles fecham o local para a noite. Walter Pitts fica trancado lá dentro.\textsuperscript{1}

Ele encontra um livro nas prateleiras que parece interessante e começa a lê-lo. Durante três dias, ele lê o livro de capa a capa.

O livro é um tratado de duas mil páginas sobre lógica formal; notavelmente, sua prova de que 1+1=2 só aparece na página 379.\textsuperscript{2} Pitts decide escrever uma carta a um dos autores – o filósofo britânico Bertrand Russell – porque acredita ter encontrado vários erros.

\bigskip

Jerry Lettvin. Pitts só se importa com lógica. Lettvin só se importa com poesia e, em segundo plano, medicina.\textsuperscript{4} Eles se tornam melhores amigos inseparáveis.

Pitts começa a frequentar o campus da Universidade de Chicago, assistindo a aulas; ele ainda não tem o diploma do ensino médio e nunca se matricula formalmente. Uma dessas aulas é dada pelo famoso lógico alemão Rudolf Carnap. Pitts entra no horário de atendimento do professor, declarando ter encontrado algumas ``falhas'' no último livro de Carnap. Cético, Carnap consulta o livro; Pitts, é claro, está certo. Eles conversam por um tempo, então Pitts sai sem dar seu nome. Carnap passa meses perguntando sobre o ``jornaleiro que conhecia lógica.''\textsuperscript{5} Eventualmente, Carnap o encontra novamente e, no que se tornaria um motivo recorrente na vida acadêmica de Pitts, torna-se seu defensor, persuadindo a universidade a lhe dar um emprego subalterno para que ele tivesse pelo menos alguma renda.

\bigskip

Agora é 1941. Lettvin – ainda um poeta em primeiro lugar, em sua própria mente – conseguiu, apesar de si mesmo, entrar na faculdade de medicina da Universidade de Illinois, e se vê trabalhando sob o comando do brilhante neurologista Warren McCulloch, recém-chegado de Yale. Um dia, Lettvin convida Pitts para conhecê-lo. Neste ponto, Lettvin tem vinte e um anos e ainda mora com os pais. Pitts tem dezessete anos e está desabrigado.\textsuperscript{6} McCulloch e sua esposa os acolhem.

Durante o ano seguinte, McCulloch volta para casa à noite e ele e Pitts – que mal é mais velho do que os próprios filhos de McCulloch – regularmente ficam acordados até depois da meia-noite conversando. Intelectualmente, eles são a equipe perfeita: o estimado neurologista de meia-carreira e o lógico prodígio. Um vive na prática – o mundo dos sistemas nervosos e neuroses – e o outro vive na teoria – o mundo dos símbolos e provas. Ambos não querem nada mais do que entender a natureza da verdade: o que ela é, e como a conhecemos. O ponto de apoio dessa busca – aquilo que se situa na intersecção perfeita de seus dois mundos díspares – é, naturalmente, o cérebro.

\bigskip

Em meses, eles escrevem um artigo juntos – o neurologista de meia-idade e o lógico adolescente. Eles o chamam de ``A Logical Calculus of Ideas Immanent in Nervous Activity.''

``Devido ao caráter de `tudo ou nada' da atividade nervosa'', eles escrevem, ``os eventos neurais e as relações entre eles podem ser tratados por meio da lógica proposicional. Constata-se que o comportamento de toda rede pode ser descrito nesses termos . . . e que para qualquer expressão lógica que satisfaça certas condições, pode-se encontrar uma rede comportando-se da forma que ela descreve.''

\bigskip

O artigo é publicado em 1943 no \textit{Bulletin of Mathematical Biophysics}. Para frustração de Lettvin, ele causa pouco impacto na comunidade da biologia.\textsuperscript{8} Para decepção de Pitts, o trabalho de neurociência dos anos 1950, notavelmente um estudo marcante do nervo óptico do sapo – feito por ninguém menos que seu melhor amigo, Jerry Lettvin – mostraria que os neurônios parecem ser muito mais confusos do que os simples circuitos de ``verdadeiro'' ou ``falso'' que ele imaginou. Talvez a lógica proposicional – seus \textit{e's}, \textit{ou's} e \textit{não's} – não fosse, em última análise, a linguagem do cérebro, ou pelo menos não de uma forma tão direta. Esse tipo de impureza entristeceu Pitts.

Mas o impacto do artigo – daquelas longas conversas noite adentro na casa de McCulloch – seria enorme, embora não inteiramente da maneira que McCulloch e Pitts imaginaram. Seria o fundamento para um campo completamente novo: o projeto de realmente construir mecanismos a partir dessas versões simplificadas de neurônios, e ver exatamente o que esses ``cérebros mecânicos'' poderiam fazer.\textsuperscript{9}